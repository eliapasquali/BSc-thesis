\chapter{Introduzione}
\label{cap:introduzione}

\section{L'azienda}

Winning Technology S.p.A è un'azienda nata nel 1987 da un'idea del suo fondatore e attuale amministratore delegato Massimo Gallotta. Wintech è un \emph{System Integrator} che opera nel settore \emph{Information and Communication Technology}, spesso abbreviato in ICT, che raggruppa tutti i servizi legati allo sviluppo di soluzioni software, hardware e progettazione ad hoc legati all'informatica e alle telecomunicazioni.

L'azienda ha sede principale a Padova ed è arrivata ad espandersi in altre tre sedi, una a Milano, una a Bassano del Grappa e l'ultima a Pordenone, contando più di 90 risorse divise tra le varie filiali.\cite{site:wtc-dati}

\begin{figure}[htbp]
    \centering
    \includegraphics[width=0.8\linewidth]{images/wtc/logo.png}
    \caption{Logo di Wintech S.p.A.}
    \label{fig:wintech-logo}
\end{figure}


\section{L'idea}

Lo scopo dello stage è quello di fornire una formazione approfondita sulle tecnologie e le metodologie relative agli strumenti di \emph{Network Detection and Response} (NDR). In particolare, lo studente avrà l'opportunità di acquisire competenze specialistiche sulla progettazione, l'implementazione e la gestione di un sistema di NDR scelto dall'azienda che verrà inserito all'interno della propria infrastruttura.

Un sistema di NDR è uno strumento che consente di individuare rapidamente i pericoli nella rete e di attuare una \emph{remediation} automatica. Monitora costantemente il traffico di rete e analizza i dati raccolti, per identificare eventuali attività sospette o pericolose. In caso di pericolo, il sistema può attivare una risposta automatica o notificare un amministratore di sistema, consentendo di rispondere in modo rapido ed efficace a potenziali attacchi.

Durante lo stage, il candidato ha lavorato alla progettazione e all'implementazione del sistema, configurando i dispositivi di sicurezza della rete, definendo le regole e le politiche di sicurezza, per poi verificarle simulando degli attacchi e analizzare i dati di monitoraggio per identificare eventuali attività sospette. L'obiettivo finale del progetto è stato quello di mettere in produzione il sistema al termine dello stage, in un contesto complesso e articolato di tre \emph{data center} collegati in rete geografica nazionale.

\section{Organizzazione del testo}

\begin{description}
    \item \hyperref[cap:azienda]{Il secondo capitolo} descrive il contesto aziendale e il suo approccio nei vari ambiti lavorativi
    \item \hyperref[cap:stage]{Il terzo capitolo} descrive in dettaglio il progetto di stage e la sua pianificazione
    \item \hyperref[cap:tecnologie]{Il quarto capitolo} raccoglie tutto ciò che è stato studiato per comprendere il contesto applicativo e le varie tecnologie utilizzate
    \item \hyperref[cap:prodotto]{Il quinto capitolo} approfondisce il prodotto e le funzionalità che offre
    \item \hyperref[cap:svolgimento]{Il sesto capitolo} descrive le attività svolte durante lo stage
    \item \hyperref[cap:conclusioni]{Il settimo capitolo} raccoglie i risultati ottenuti e alcune considerazioni personali
\end{description}

Riguardo la stesura del testo, relativamente al documento, sono state adottate le seguenti convenzioni tipografiche:
\begin{itemize}
	\item I termini tecnici e in lingua straniera sono scritti in \emph{corsivo}, mentre pezzi di codice invece verranno scritti in carattere \texttt{monospaziato};
	\item Le figure e le tabelle possiedono una descrizione e una numerazione progressiva legata al capitolo di appartenenza.
\end{itemize}

Data la natura del progetto, sono stati raccolti e analizzati dati riservati aziendali e dei clienti, per questo motivo, dalle immagini sono state rimosse tutte le componenti che potrebbero rivelare informazioni sensibili.

Per alleggerire il documento, sono stati inserite delle appendici che contengono informazioni aggiuntive e approfondimenti su alcuni argomenti trattati, come ad esempio la configurazione di alcuni dispositivi di rete in \autoref{appendix:schemi} e alcuni attacchi e mitigazioni in \autoref{appendix:security}.
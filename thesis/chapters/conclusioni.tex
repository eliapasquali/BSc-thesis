\chapter{Conclusioni}
\label{cap:conclusioni}

\section{Obbiettivi raggiunti}

Rispetto agli obiettivi descritti nella \autoref{sez:obiettivi}:

\begin{itemize}
    \item Per la categoria \textbf{Obbligatori} il \emph{deploy} e la configurazione del prodotto è avvenuta. Il sistema è stato testato e le \emph{remediation} analizzate e definite, nei limiti dell'integrazione con sistemi di terze parti. Tutte le operazioni sono state documentate e sono state scritte le procedure per la gestione del sistema.
    \item Per la categoria \textbf{Desiderabili} invece non è stato possibile integrare ulteriori apparati dell'infrastruttura di Wintech oltre all'\emph{EDR} di Sangfor.
    \item Sempre per gli stessi motivi, anche la categoria \textbf{Facoltativi} ha risentito della mancanza di integrazioni con sistemi di terze parti. Inoltre, tutti i test effettuati sono stati rilevati, confermando la capacità del sistema di rilevare anche attacchi interni alla rete.
\end{itemize}

\section{Segnalazioni e miglioramenti proposti}

Al termine del percorso di stage ho raccolto tutte i problemi riscontrati, come ad esempio la gestione dei parametri multipli, e i miglioramenti possibili, come la definizione di \emph{policy} tramite linguaggio di \emph{scripting}. Tutto questo è stato discusso con il tutor e poi riportato al fornitore tramite il \emph{team} italiano, che ha ringraziato e poi provveduto a inoltrare il resoconto dell'esperienza al \emph{team} di sviluppo.

\section{Valutazione personale}

\subsection{Conoscenze tecniche acquisite}

La scelta di un progetto in questo ambito mi ha permesso di approfondire il campo delle reti dal lato pratico, cosa che per ovvie ragioni è difficile da vedere durante il corso di studi. Lavorare con strumentazione reale e di livello \emph{enterprise} mi ha permesso di vedere come vengono gestiti i sistemi in un contesto aziendale, espandendo la mia visione che si limitava a simulazioni e piccoli progetti personali con \emph{container} e \emph{virtual machine}.

Ascoltando consigli di colleghi e tutor, ho potuto capire come spesso in un contesto del genere è necessario scegliere un \emph{trade-off} tra la soluzione teorica più elegante e le limitazioni imposte dall'ambiente in cui un particolare sistema o strumento andrà a operare.

Aver utilizzato uno strumento di NDR mi ha permesso di capire come funzionano i sistemi di rilevazione all'avanguardia e come sia necessario un continuo aggiornamento per stare al passo con le minacce sempre più sofisticate. 

\subsection{Conoscenze personali acquisite}

Un percorso di questo tipo mi ha permesso di inserirmi per un periodo di tempo abbastanza importante nel mondo del lavoro. Questo mi ha fatto capire come funziona un'azienda, come si lavora in un \emph{team} e come si gestiscono le scadenze e le priorità.

A differenza dei progetti durante il corso di studi, tra amici, con scadenze molto più flessibili e con un margine di errore più ampio, in un contesto aziendale è necessario rispettare le richieste e le tempistiche, per non compromettere il lavoro di altri colleghi.

Inoltre, ho avuto occasione di interagire con clienti e fornitori e questo mi ha permesso di capire come gestire le relazioni con le persone in contesti lavorativi, in modo da poter ottenere il massimo risultato da ogni interazione.

\subsection{Gap tra Università e mondo del lavoro}

In un corso di studi come quello di Informatica è impossibile coprire tutti gli argomenti e le tecnologie che nascono e si evolvono ogni giorno. Tuttavia la base fornita è estremamente solida e permette di affrontare un problema o lo studio di nuove tecnologie in modo autonomo.

Nel mio caso particolare, come detto prima, ho potuto toccare con mano un campo che non viene trattato durante il corso di studi nella pratica, ma solo nella teoria. Dal lato della gestione del progetto invece forse solo il progetto di Ingegneria del Software riesce a dare un'idea di scadenze e tempistiche da rispettare.